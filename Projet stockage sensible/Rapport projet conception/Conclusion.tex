
\chapter*{Conclusion}
\addcontentsline{toc}{chapter}{\protect\numberline{}Conclusion}



Ce projet a permit de réaliser un programme permettant le dimensionnement d'une unité de stockage sensible en régime in-stationnaire. Nous avons rencontré énormément de problèmes numériques pour réaliser ce dernier. De ce fait, nous avons dut nous contenter de simulations pour de petites dimensions. Grâce à ces dernières, nous avons fait de nombreuses analyses de sensibilités permettant de mettre en avant les subtilités de ce type de stockage.\\

Le projet avait pour objectif de dimensionner une unité de stockage pour la central de Huntorf. De ce fait, nous avons fait des calculs préliminaires pour constater l'avantage de récupérer la chaleur des compresseurs sur ce type d'installation. De plus, nous avons évalué un ordre de grandeur de l'investissement possible pour installer notre système. Nous avons par la suite estimé la température en sortie des compresseurs pour connaitre les caractéristiques de l'air en entrée de notre échangeur. Ainsi, nous avons posé les données essentielles à notre simulation. Cependant, la complexité du programme nous a empêché de la réalisée, puisque les dimensions nécessaire étaient trop grande, même pour le cas idéal. Ceci dit, la récente amélioration du programme devrait permettre de faire cela. Nous ferons en sorte de vous présenter les résultats lors de la soutenance oral. Un petit sursis de temps nous aurait également permit de lui attribuer les quelques astuces programmé dans sa version précédente, mais surtout les calculs pour le dimensionnement en régime régime nominal. Bien que l'objectif n'a pas été parfaitement accomplit, il faut tout de même noter que nous avons écrit des programmes de plus de milles lignes et que nous avons fini par aboutir sur une version très puissante (donc très rapide). De plus, il peut très bien être utilisé pour simuler n'importe qu'elle type d'échangeur, de même configuration, en régime in-stationnaire. Réalisé ce que nous avons fait dans un laps de temps aussi serré était loin d'être facile et par conséquent nous somme amplement satisfait de ce que nous avons accomplit. 



